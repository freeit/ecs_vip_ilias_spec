\documentclass[12pt,a4paper]{article}
\usepackage[german]{babel}
\usepackage{textcomp}
\usepackage[utf8x]{inputenc}
\usepackage[T1]{fontenc}
\usepackage{color}
\usepackage{graphicx}
\usepackage{chngpage}
\usepackage{ifthen}
\usepackage{gitinfo}
%\enabledraftstandard
\usepackage[firstpage]{draftwatermark}
\usepackage[hidelinks]{hyperref}
\usepackage{url}
\usepackage{xcolor,soul,lipsum}

\newcommand{\myul}[1]{\ul{#1}}

\SetWatermarkText{ENTWURF}
\SetWatermarkScale{4}
\SetWatermarkAngle{50}
\SetWatermarkLightness{0.90}

\sloppy
%\hoffset-3mm
\parindent0mm

\newcommand{\da}{DA}
\newcommand{\sa}{SA}
\newcommand{\ecs}{ECS}
\newcommand{\ilias}{ILIAS}
\newcommand{\button}[1]{\fbox{\texttt{#1}}}
\newcommand{\jsToAppletComUrlA}{\mbox{http://stackoverflow.com/questions/7278626/javascript-to-java-applet-communication}}
\newcommand{\jsToAppletComUrlB}{\mbox{http://docs.oracle.com/javase/tutorial/deployment/applet/invokingAppletMethodsFromJavaScript.html}}
\newcommand{\appletToDom}{\mbox{http://docs.oracle.com/javase/tutorial/deployment/applet/manipulatingDOMFromApplet.html}}

\begin{document}
\title{VIP-LMS Anbindung}
\author{Heiko Bernl"ohr\\FreeIT.de \and Thomas Richter\\Uni Stgt. \and David Boehringer\\Uni Stgt. \and Per Pascal Grube\\Uni Stgt. \and Stephan Rudlof\\Uni Stgt. }
\maketitle
\begin{center}
%  Version: 2.0
  Version: \gitVtag\\
%  Commit-id: \gitAbbrevHash
\end{center}
\begin{abstract}
  Maximale Integration von VIP in Ilias. Applet $\leftrightarrow$ Backend
  Kommunikation wie gehabt über ECS.
\end{abstract}

\vspace{\fill}
\begin{center}
\includegraphics[scale=1.0]{vip_lms}
\end{center}
\vspace{\fill}

\newpage

\section{Dozentenapplet Workflows}
\subsection{Erstellen und erweitern von Aufgaben}
\begin{itemize}
  \item Das Applet wird aus einem Aufgabenpool von Ilias heraus gestartet,
    entweder zur Erstellung einer neuen oder zum Überarbeitung einer bereits
    bestehenden Aufgabe (exercise).
  \item Der Dozent bearbeitet/erstellt seine Aufgabe mit dem Applet und kann
    diese auch jederzeit durch Angabe einer Lösung durch drücken von
    \button{berechnen}\footnote{Wird von Applet implementiert.} testweise
    berechnen lassen.
  \item Der Dozent kann seine Aufgabe jederzeit durch drücken von
    \button{speichern}\footnote{Wird von Ilias implementiert (ExSave:Button).}
    sichern, das Applet verlassen/beenden und zu einem späteren Zeitpunkt durch
    erneute Auswahl der Aufgabe in Ilias wieder bearbeiten.
\end{itemize}
Siehe Sequenzdiagramm auf Seite \pageref{seq:la_create_exercise}, Abbildung \ref{seq:la_create_exercise}.

\subsection{Korrektur von Aufgaben}
Siehe Sequenzdiagramm auf Seite \pageref{seq:la_eval_exercise}, Abbildung \ref{seq:la_eval_exercise}.

\section{Studentenapplet Workflows}
\subsection{Lösung abgeben}
\begin{itemize}
  \item Das Applet wird beim Bearbeiten einer VIP-Aufgabe innerhalb eines Ilias-Tests
    oder als alleinstehende Aufgabe gestartet.
  \item Der Student kann jederzeit eine Berechnung seiner eingegebenen Lösung
    durch drücken von \button{berechnen}\footnote{Wird von Applet
    implementiert.} anfordern.
  \item Der Student gibt seine eingegebene Lösung durch drücken von
    \button{weiter}\footnote{Wird von Ilias implementiert
    (SolutionSave:Button).} ab. Sollte bis dahin noch keine oder keine aktuelle
    Berechnung der Aufgabe vorliegen, weist das Applet daraufhin, eine solche
    durchzuführen.
\end{itemize}
Siehe Sequenzdiagramm auf Seite \pageref{seq:sa_create_solution_seq}, Abbildung \ref{seq:sa_create_solution_seq}.

\begin{figure}[p]
  \includegraphics[scale=0.8]{la_create_exercise_seq}
  \caption{\label{seq:la_create_exercise}Erstellen/überarbeiten einer Aufgabe}
\end{figure}

\begin{figure}[p]
  \includegraphics[scale=0.8]{la_eval_exercise_seq}
  \caption{\label{seq:la_eval_exercise}Evaluieren einer studentischen Aufgabe durch den Dozenten}
\end{figure}

\begin{figure}[p]
  \includegraphics[scale=0.8]{sa_create_solution_seq}
  \caption{\label{seq:sa_create_solution_seq}Lösungsabgabe einer aus Ilias
  heraus gestarteten Aufgabenbearbeitung.}
\end{figure}

\section{Ressourcen}
\subsection{/sys/subparticipants}
Ein Subparticipant stellt einen untergeordneten Participanten dar, der am \ecs\
durch einen Standardparticipanten (statisch eingetragener Participant)
dynamisch, über die \texttt{/sys/subparticipants} Ressource,
angemeldet/registriert wird. Im Unterschied zu einem klassischen anonymen
Participanten darf er nicht nur der \textit{public} community angehören,
sondern kann durch den registrierenden Participanten jeder Community zugeteilt
werden, welchen der registrierende Participant selbst auch angehört. Die
\texttt{/sys/subparticipants} Ressource sieht folgendermaßen aus:
\begin{verbatim}
{
  "realm": "<any context>",           #optional, maybe a session id
  "communities": ["<cid|name>", ...], #optional, default:"public"
  "community_selfrouting": <boolean>, #optional, default:false
  "events": <boolean>,                #optional, default:true
  "cookie": <cookie>                  #wird von ECS erzeugt
}
\end{verbatim}

\begin{description}
  \item[GET] Liefert realm, communities, community\_selfrouting, events und cookie (Statuscode 200).
  \item[POST] Erzeugt neuen Subparticipanten (Statuscode 201) und liefert im \textit{Location:} Header die neue Representations-URL des Subparticipanten. Mögliche Aufrufparameter: realm, communities, community\_selfrouting, events.
  \item[DELETE] Löscht einen Subparticipanten (Statuscode 200). Mögliche Aufrufparameter: realm, communities, community\_selfrouting, events.
  \item[PUT] Erlaubt das Ändern der Subparticipanten-Attribute: realm, communities, community\_selfrouting, events. Wie immer wird bei der PUT Operation die komplette Representation (alle Attribute) erneuert. Nicht angegebene Attribute erhalten also den Defaultwert. Liefert Statuscode 200.
\end{description}

\section{Sicherheit}
\subsection{Zertifikate}
Ein Subparticipant wird entgegen normaler Participanten nicht mit einem
Clientzertifikat ausgestattet und kann somit auch keiner Peer-Authentifikation
unterzogen werden, wird aber stattdessen vom generierenden Participanten mit
einem individuellen Cookie versorgt, so daß es genügt, den Subparticipanten mit
dem entsprechenden CA-Root-Zertifikat auszustatten. Dieses CA-Root-Zertifikat
ist öffentlicher Natur und kann bedenkenlos ins Applet integriert oder auch
dynamisch\footnote{CA-Root-Cert von FreeIT:
\url{http://FreeIT.de/pub/freeit-root-certificate.crt.pem}} geladen werden.



\clearpage
\begin{appendix}
\section{Browser $\rightarrow$ Applet Kommunikation}\label{btoacom}
\begin{itemize}
  \item \href{\jsToAppletComUrlA}{\myul{Javascript to Java Applet communication}}\footnote{\jsToAppletComUrlA}
  \item \href{\jsToAppletComUrlB}{\myul{Invoking Applet Methods From JavaScript Code}}\footnote{\jsToAppletComUrlB}
\end{itemize}
\section{Applet $\rightarrow$ Browser Kommunikation}
\begin{itemize}
  \item \href{\appletToDom}{\myul{Manipulating DOM of Applet's Web Page}}\footnote{\appletToDom}
\end{itemize}
\end{appendix}
\end{document}
