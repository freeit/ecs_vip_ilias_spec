\documentclass[12pt,a4paper]{article}
\usepackage[german]{babel}
\usepackage{textcomp,marvosym}
\usepackage[utf8x]{inputenc}
\usepackage[T1]{fontenc}
\usepackage{color}
\usepackage{graphicx}
\usepackage{chngpage}
\usepackage{ifthen}
\usepackage{gitinfo}
%\enabledraftstandard
\usepackage[firstpage]{draftwatermark}
\usepackage[hidelinks]{hyperref}
\usepackage{url}
\usepackage{xcolor,soul,lipsum}

\newcommand{\myul}[1]{\ul{#1}}

\SetWatermarkText{ENTWURF}
\SetWatermarkScale{4}
\SetWatermarkAngle{50}
\SetWatermarkLightness{0.90}

\sloppy
%\hoffset-3mm
\parindent0mm

\newcommand{\da}{DA}
\newcommand{\sa}{SA}
\newcommand{\ecs}{ECS}
\newcommand{\ilias}{ILIAS}
\newcommand{\button}[1]{\fbox{\texttt{#1}}}
\newcommand{\jsToAppletComUrlA}{\mbox{http://stackoverflow.com/questions/7278626/javascript-to-java-applet-communication}}
\newcommand{\jsToAppletComUrlB}{\mbox{http://docs.oracle.com/javase/tutorial/deployment/applet/invokingAppletMethodsFromJavaScript.html}}
\newcommand{\appletToDom}{\mbox{http://docs.oracle.com/javase/tutorial/deployment/applet/manipulatingDOMFromApplet.html}}

\begin{document}
\title{VIP-LMS Anbindung}
\author{Heiko Bernl"ohr, Per Pascal Grube, Thomas Richter, David Boehringer}
\maketitle
\begin{center}
%  Version: 2.0
  Version: \gitVtag\\
%  Commit-id: \gitAbbrevHash
\end{center}
\begin{abstract}
  Maximale Integration von VIP in Ilias. Applet $\leftrightarrow$ Backend
  Kommunikation wie gehabt über ECS.
\end{abstract}

\vspace{\fill}
\begin{center}
\includegraphics[scale=1.0]{vip_lms}
\end{center}
\vspace{\fill}

\newpage

\section{Dozentenapplet}
\subsection{Workflow}
\begin{itemize}
  \item Das Applet wird aus einem Aufgabenpool von Ilias heraus gestartet,
    entweder zur Erstellung einer neuen oder zum Überarbeitung einer bereits
    bestehenden Aufgabe (exercise).
  \item Die Aufgabe (exercise) wird als \textit{hidden field} im HTML
    Dokument eingebettet, so da"s das Applet diese dort entnehmen kann.
  \item Beim Starten des Applets werden folgende Parameter mitgegeben:
    \begin{itemize}
      \item ECS-URL
      \item EcsAuthId
      \item Ilias mid
      \item Computation-Backend mid
    \end{itemize}
  \item Der Dozent bearbeitet/erstellt seine Aufgabe mit dem Applet und kann
    diese auch jederzeit durch Angabe einer Lösung durch drücken von
    \button{berechnen}\footnote{wird von Applet implementiert} testweise
    berechnen lassen.
  \item Der Dozent kann seine Aufgabe jederzeit durch drücken von
    \button{speichern}\footnote{wird von Ilias implementiert} sichern, das
    Applet verlassen/beenden und zu einem späteren Zeitpunkt durch erneute
    Auswahl der Aufgabe in Ilias wieder bearbeiten. Dabei wird die
    neue/gänderte Aufgabe (exercise) ebenfalls über ein \textit{hidden field}
    im HTML Dokument (ExSave:Button Kontext) ans Ilias zurückgegeben.
\end{itemize}
Siehe Sequenzdiagramm auf Seite \pageref{seq:la_create_exercise}, Abbildung \ref{seq:la_create_exercise}.

\section{Studentenapplet}
\subsection{Workflow}
\begin{itemize}
  \item Das Applet wird beim Bearbeiten einer VIP-Aufgabe innerhalb eines Ilias-Tests
    oder als alleinstehende Aufgabe gestartet.
  \item Die Aufgabe (exercise) wird als \textit{hidden field} im HTML
    Dokument eingebettet, so da"s das Applet diese dort entnehmen kann.
  \item Beim Starten des Applets werden folgende Parameter mitgegeben:
    \begin{itemize}
      \item ECS-URL
      \item EcsAuthId
      \item Ilias mid
      \item Computation-Backend mid
    \end{itemize}
  \item Der Student kann jederzeit eine Berechnung seiner eingegebenen Lösung
    durch drücken von\addtocounter{footnote}{-2}
    \button{berechnen}\footnotemark anfordern.
  \item Der Student gibt seine eingegebene Lösung durch drücken von
    \button{weiter}\footnotemark ab, wobei \textit{solutions} und
    \textit{results} als \textit{hidden field} im HTML Dokument eingebettet
    werden. Sollte bis dahin noch keine oder keine aktuelle Berechnung der
    Aufgabe vorliegen, weist das Applet daraufhin eine solche durchzuführen.
  \item Applet beendet sich.
\end{itemize}
Siehe Sequenzdiagramm auf Seite \pageref{seq:sa_create_solution_seq}, Abbildung \ref{seq:sa_create_solution_seq}.

\begin{figure}[p]
  \includegraphics[scale=0.8]{la_create_exercise_seq}
  \caption{\label{seq:la_create_exercise}Erstellen/überarbeiten einer Aufgabe}
\end{figure}

\begin{figure}[p]
  \includegraphics[scale=0.8]{sa_create_solution_seq}
  \caption{\label{seq:sa_create_solution_seq}Lösungsabgabe einer aus Ilias
  heraus gestarteten Aufgabenbearbeitung.}
\end{figure}

\section{Ressourcen}
\subsection{/sys/subparticipants}
Ein Subparticipant stellt einen untergeordneten Participanten dar, der am \ecs\
durch einen Standardparticipanten (statisch eingetragener Participant)
dynamisch, über die \texttt{/sys/subparticipants} Ressource,
angemeldet/registriert wird. Im Unterschied zu einem klassischen anonymen
Participanten darf er nicht nur der \textit{public} community angehören,
sondern kann durch den registrierenden Participanten jeder Community zugeteilt
werden, welchen der registrierende Participant selbst auch angehört. Die
\texttt{/sys/subparticipants} Ressource sieht folgendermaßen aus:
\begin{verbatim}
{
  "mapping_id": "<id>",               #maybe a session id
  "communities": ["<cid|name>", ...], #optional, default:"public"
  "community_selfrouting": <boolean>, #optional, default:false
  "events": <boolean>                 #optional, default:true
}
\end{verbatim}

\clearpage
\begin{appendix}
\section{Browser $\rightarrow$ Applet Kommunikation}\label{btoacom}
\begin{itemize}
  \item \href{\jsToAppletComUrlA}{\myul{Javascript to Java Applet communication}}\footnote{\jsToAppletComUrlA}
  \item \href{\jsToAppletComUrlB}{\myul{Invoking Applet Methods From JavaScript Code}}\footnote{\jsToAppletComUrlB}
\end{itemize}
\section{Applet $\rightarrow$ Browser Kommunikation}
\begin{itemize}
  \item \href{\appletToDom}{\myul{Manipulating DOM of Applet's Web Page}}\footnote{\appletToDom}
\end{itemize}
\end{appendix}
\end{document}
